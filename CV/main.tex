%% start of file `template.tex'.
%% Copyright 2006-2013 Xavier Danaux (xdanaux@gmail.com).
%
% This work may be distributed and/or modified under the
% conditions of the LaTeX Project Public License version 1.3c,
% available at http://www.latex-project.org/lppl/.


\documentclass[11pt,a4paper,sans]{moderncv}        % possible options include font size ('10pt', '11pt' and '12pt'), paper size ('a4paper', 'letterpaper', 'a5paper', 'legalpaper', 'executivepaper' and 'landscape') and font family ('sans' and 'roman')

% moderncv themes
\moderncvstyle{classic}                            % style options are 'casual' (default), 'classic', 'oldstyle' and 'banking'
\moderncvcolor{green}                              % color options 'blue' (default), 'orange', 'green', 'red', 'purple', 'grey' and 'black'
%\renewcommand{\familydefault}{\sfdefault}         % to set the default font; use '\sfdefault' for the default sans serif font, '\rmdefault' for the default roman one, or any tex font name
%\nopagenumbers{}                                  % uncomment to suppress automatic page numbering for CVs longer than one page

% character encoding
\usepackage[utf8]{inputenc}                       % if you are not using xelatex ou lualatex, replace by the encoding you are using
%\usepackage{CJKutf8}                              % if you need to use CJK to typeset your resume in Chinese, Japanese or Korean

% adjust the page margins
\usepackage[scale=0.75]{geometry}
%\setlength{\hintscolumnwidth}{3cm}                % if you want to change the width of the column with the dates
%\setlength{\makecvtitlenamewidth}{10cm}           % for the 'classic' style, if you want to force the width allocated to your name and avoid line breaks. be careful though, the length is normally calculated to avoid any overlap with your personal info; use this at your own typographical risks...

% personal data
\name{Chris}{Denniston}
%\title{Resumé title}                               % optional, remove / comment the line if not wanted
\address{2650 Van Buren Pl.}{90007}{United States}% optional, remove / comment the line if not wanted; the "postcode city" and and "country" arguments can be omitted or provided empty
\phone[mobile]{+1~(913)~908~6148}                   % optional, remove / comment the line if not wanted
\email{cdennist@usc.edu}                               % optional, remove / comment the line if not wanted

%----------------------------------------------------------------------------------


%            content
%----------------------------------------------------------------------------------
\begin{document}
%-----       letter       ---------------------------------------------------------
% recipient data
\recipient{MBARI, Human Resources}{7700 Sandholt Road\\ Moss Landing, CA 95039}
\date{January 29, 2019}
\opening{Dear Sir or Madam,}
\closing{Thank you for your consideration,}
\makelettertitle

I am excited to apply for the position of research engineer at the Montery Bay Aquarium Research Institute. I am highly motivated to work on underwater systems as well as machine learning systems, especially when they have impact on the environment. 

During my role as student researcher I have worked on adaptive sampling for underwater autonomus vehicles. I coauthored a paper which was published in IEEE AUV 2018 on using adaptive sampling techniques to plan around hazardous areas. During that time I spoke with many MBARI employees who were also presenting. I am excited for the work MBARI is doing in field with underwater vision and filtration sampling. 

I have also worked on vision based projects during an internship at MITRE in San Diego. I helped build a system to identify specific aircraft from aerial imagery using deep learning techniques. This was a technical challenge because there was very little known imagery of these aircraft as well as an organizational challenge because five different teams were working on different aspects of the project. 


\makeletterclosing

\end{document}


%% end of file `template.tex'.
